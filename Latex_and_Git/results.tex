\section{Results}
\subsection{Emotional Awareness and Expressive Reluctance}
\subsubsection{Child-Report } The analyses revealed no significant interaction between time x subscales, F(2, 43) = .10, p = .91, ηp2 = .01. Results indicated no significant main effect of time, F(2, 43) = 1.62, p = .21, ηp2 = .07. However, there was a significant main effect for subscale type, F(1, 44) = 20.08, p < .001, ηp2 = .31, indicating differences in levels of poor awareness and expressive reluctance, independent of time. 
To further investigate these effects, a paired-samples t-test was conducted to compare scores on the poor awareness and expressive reluctance subscales. Results indicated that the average scores on the expressive reluctance subscales (M = 22.94, SD = 6.26) were significantly higher than average scores on the poor awareness subscales (M = 19.62, SD = 7.28), t(45) = 4.48, p < .001. See Table  for child-report means and standard deviations. 
 (see Figure~\ref{fig:TikZmodel}).

\begin{figure}[htbp]
\centering
\resizebox{\textwidth}{!}{%
\begin{tikzpicture}[
  node distance=18mm,
  box/.style={draw, rounded corners=2mm, align=center, inner sep=5pt, minimum width=3.2cm},
  >={Latex}
]
% Nodes
\node[box, fill=gray!10] (int) {Modified\\ TINT:WSA\\ (6 sessions)};
\node[box, right=18mm of int, fill=gray!10] (skills) {Emotion Competencies\\[1pt]
\footnotesize Awareness $\uparrow$\\ Expression reluctance $\downarrow$\\ Inhibition $\downarrow$\\ Dysregulation $\downarrow$\\ Coping $\uparrow$};
\node[box, right=18mm of skills, fill=gray!10] (symp) {Psychopathology\\[1pt]
\footnotesize Internalizing $\downarrow$\\ Externalizing $\downarrow$};

% Arrows without labels
\draw[->, line width=0.9pt] (int) -- (skills);
\draw[->, line width=0.9pt] (skills) -- (symp);

% Moderators / emotion types
\node[below=14mm of skills, align=center, font=\footnotesize] (emo)
{Emotion types: Anger \quad Sadness \quad Worry};
\draw[->, dashed] (emo.north) -- (skills.south);

% Notes (timepoints)
\node[below=14mm of int, font=\footnotesize, align=center] (tpts)
{Assessments: T1 (pre), T2 (post), T3 (1-mo follow-up)};
\end{tikzpicture}%
}
\caption{Conceptual model: The modified TINT:WSA aims to enhance adolescents' emotion competencies, which in turn reduce internalizing and externalizing symptoms. }
\label{fig:conceptual_model}
\end{figure}

In other cases, you might have an image file you want to bring into your document.  That is pretty easy to do as well as you can see in Figure \ref{fig:gearhead}.

\begin{figure}[h!] 
\centering
\includegraphics[width=0.3\textwidth]{frog.jpeg}
\caption{\label{fig:gearhead}This is a another figure caption.}
\end{figure}

Sometimes You may also want to display some tabular data in order to illustrate patterns in your data. Fortunately, You can do this right in \LaTeX.

\begin{table}[htbp]
\centering
\caption{Means, Standard Deviations, and RM-ANOVAs for the Child-Report Outcomes}
\label{tab:child_outcomes}
\begin{tabular}{lccccccccc}
\toprule
\multirow{2}{*}{Measure} & \multicolumn{2}{c}{T1} & \multicolumn{2}{c}{T2} & \multicolumn{2}{c}{T3} & \multirow{2}{*}{F} & \multirow{2}{*}{p} & \multirow{2}{*}{$\eta_p^2$} \\
 & M & SD & M & SD & M & SD & & & \\
\midrule
\textbf{EESC} & & & & & & & & & \\
\quad PA & 20.27 & 8.00 & 19.31 & 7.62 & 19.29 & 7.72 & 1.01 & .37 & .05 \\
\quad ER & 23.60 & 6.24 & 22.78 & 7.23 & 22.44 & 6.85 & 1.12 & .34 & .06 \\
\textbf{CAMS} & & & & & & & & & \\
\quad Inh. & 1.88 & 0.59 & 1.84 & 0.57 & 1.91 & 0.57 & 0.63 & .51 & .01 \\
\quad Dys. & 1.61 & 0.45 & 1.70 & 0.59 & 1.67 & 0.64 & 0.91 & .40 & .02 \\
\quad Cop. & 2.11 & 0.53 & 2.28 & 0.48 & 2.26 & 0.56 & 3.53 & .04 & .14 \\
\bottomrule
\end{tabular}
\end{table}

