
Adolescence is a developmental period characterized by rapid biological, cognitive, and socio-emotional changes that shape mental health trajectories. During this time, youth face increased vulnerability to the onset of psychological disorders, in part due to heightened emotional reactivity and evolving social-cognitive capacities \citep{jaworska2015adolescent}. Emotional awareness, or the ability to identify, differentiate, and understand one’s emotional experiences, has been shown to be a transdiagnostic predictor of internalizing symptoms such as depression and anxiety \citep{kranzler2016emotional}. Deficits in these competencies may also exacerbate externalizing difficulties by undermining youths’ capacity for effective regulation.

Research highlights the importance of emotion regulation in promoting adaptive development and reducing risk for psychopathology. Foundational work suggests that effective regulation of emotions is associated with positive social and psychological adjustment, while poor regulation predicts maladaptive outcomes \citep{eisenberg2001relations}. Furthermore, studies of emotion processing indicate that attentional mechanisms influence how adolescents encode and respond to emotional information, linking cognitive processes with socio-emotional functioning \citep{brenner2014role}.

Intervention research has increasingly focused on targeting these emotion skills. For example, the Tuning in to Teens (TINT) program aims to improve parents’ emotion coaching, which has been associated with reductions in adolescent internalizing difficulties \citep{kehoe2014tuning}. Building on this framework, the current study evaluates a modified version of TINT adapted for early adolescents, with a focus on strengthening emotion competencies to reduce internalizing and externalizing symptoms.

We advance two primary hypotheses. First, we expect adolescents will demonstrate improvements in emotional competence—specifically emotional awareness, reduced inhibition and dysregulation, and greater use of adaptive coping strategies—following participation in the program. Second, we expect reductions in internalizing and externalizing symptoms across the same period. Together, these hypotheses test whether directly targeting emotion skills can improve psychological functioning in early adolescence.

