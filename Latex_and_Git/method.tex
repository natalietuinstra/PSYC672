\section{Method}
\subsection{Participants}
Participants were recruited from the Williamsburg, Virginia community and were required to be fluent in English and enrolled in 5th, 6th, or 7th grade. The final sample consisted of 45 adolescents (Mage= 11.79 years, SD = 11.41 months; n = 26 girls, 57.8\%) in 5th (n = 14, 31.1\%.), 6th (n = 18, 40\%), and 7th (n = 13, 28.9\%) grades. The sample was predominantly White (n = 34, 75.6\%; Other/Multiracial, n = 7, 15.6\%; Asian, n = 3, 6.7\%; Black, n = 1, 2.2\%). 
In addition to adolescent participants, their primary caregivers provided data (Mage= 41.34 years, SD = 9.19 years; 36 mothers, 9 fathers). However, six caregivers did not complete all three time points, resulting in a final caregiver sample of 39 (Mage= 41.08 years, SD = 9.56 years; 32 mothers, 7 fathers). Specifically, one caregiver did not complete T2 questionnaires, while the remaining five did not complete T3. 
\subsection{Measures}
\subsubsection{Emotion Measures}
\textbf{Emotion Expression Scale for Children (EESC).} The EESC  is a 16-item questionnaire that uses a 5-point response scale (1 = not at all true to 5 = extremely true). The questionnaire is comprised of two subscales of eight items each. The Poor Awareness subscale assesses children’s difficulty in identifying emotions (e.g., “I often do not know how I am feeling.”). Higher scores indicate greater difficulty in emotional awareness. The Expressive Reluctance subscale assesses hesitancy in expressing emotions (e.g., “I prefer to keep my emotions to myself.”). Higher scores suggest greater reluctance to express emotions. The EESC has demonstrated high internal consistency in a community sample of early adolescents, and has shown validity through correlations with measures of emotion management. In the present study, reliability for these child-report scales were acceptable across all three time points, with internal consistencies ranging from .77 to .89. 
A parent-report version of the EESC  was also administered to gain parent perspectives on their child’s emotional awareness and expression. Parents completed the same 16-item measure, rating their child’s tendencies using the identical 5-point response scale. The EESC-P has demonstrated strong internal consistency and predictive validity in prior research (Kerns et al., 2014). In the current study, reliability for these parent-report scales was acceptable, with internal consistencies ranging from .69 to .84.
