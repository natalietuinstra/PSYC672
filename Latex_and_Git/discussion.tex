\section{Discussion}

The purpose of the present study was to evaluate the outcomes of a modified version of the TINT:WSA program, adapted to teach emotion skills directly to early adolescents rather than solely to parents. This study extends prior research in several ways. Unlike previous implementations that primarily targeted parents with the expectation that they would transfer learned skills to their children through emotion socialization, this study provided direct instruction to adolescents regarding their emotional awareness, emotional understanding, and emotion regulation strategies. The program also addressed the role of emotions within the context of peer and friendship interactions, a particularly salient domain for early adolescents . 

Sometimes the discussion section may even include additional figures.  If the figures are small enough and you don't want them to take up the full line, you can always use the \texttt{wrapfigure} environment.

\begin{wrapfigure}{l}{0.4\textwidth}     \centering       
\includegraphics[width=0.25\textwidth]{brain-lateral.png}
\caption{\label{fig:latbrain} Yet another figure caption here.}
\end{wrapfigure}

Figure~\ref{fig:gearhead} was also a great example for the Results right? But what if you have a small figure and you just want to wrap the text neatly around it?  Never fear....the ``wrapfig'' package is here! The wrapfig environment allows you to neatly wrap text around your figures just like the journals do. \blindtext
